% Template:     Template Auxiliar LaTeX
% Documento:    Archivo principal
% Versión:      7.1.5 (15/03/2021)
% Codificación: UTF-8
%
% Autor: Pablo Pizarro R.
%        Facultad de Ciencias Físicas y Matemáticas
%        Universidad de Chile
%        pablo@ppizarror.com
%
% Sitio web:    [https://latex.ppizarror.com/auxiliares]
% Licencia MIT: [https://opensource.org/licenses/MIT]

% CREACIÓN DEL DOCUMENTO
\documentclass[letterpaper, oneside]{article}

% INFORMACIÓN DEL DOCUMENTO
\def\tituloauxiliar {How to Find a Quasar?}
\def\temaatratar {Tutorial \#4}

\def\autordeldocumento {Nicolás Guerra-Varas}
\def\nombredelcurso {Introduction to AGN}
\def\codigodelcurso {MASS}

\def\nombreuniversidad {University of Belgrade}
\def\nombrefacultad {Faculty of Mathematics}
\def\departamentouniversidad {Departament of Astronomy}
\def\imagendepartamento {img/mass}
\def\imagendepartamentoparams {height=1.9cm}
\def\localizacionuniversidad {Belgrade, Serbia}

% EQUIPO DOCENTE
\def\equipodocente {
	\textbf{Nicolás Guerra-Varas} \\
	Professor Dragana Ili\'c \\
	Tutor Isidora Jankov \\
    \today \\
}

% IMPORTACIÓN DEL TEMPLATE
\input{template}

% INICIO DE PÁGINAS
\begin{document}
	
% CONFIGURACIÓN DE PÁGINA Y ENCABEZADOS
\templatePagecfg

% ----------------------------------------------------------------------------

The goal of this tutorial was to ...


\section{Part I}

\subsection{Getting the Data}

First of all, I downloaded the data from SDSS DR18 \cite{sdss_dr18} using the following SQL query:

\begin{sourcecode}[\label{sdss_query}]{sql}{SQL Query}
SELECT s.specobjid, s.plate, s.mjd, s.fiberID, s.subclass, s.z, 
g.oiii_5007_flux, g.oiii_5007_flux_err, 
g.h_alpha_flux, g.h_alpha_flux_err, 
g.h_beta_flux, g.h_beta_flux_err, 
g.nii_6584_flux, g.nii_6584_flux_err, 
W.w1mpro as w1, W.w2mpro as w2, W.w3mpro as w3

FROM GalSpecLine AS g 
JOIN SpecObj AS s ON g.specobjid = s.specobjid
JOIN wise_xmatch AS x ON s.bestobjid = x.sdss_objid
JOIN wise_allsky AS w ON w.cntr = x.wise_cntr

WHERE
(s.class = 'QSO' or s.class = 'GALAXY')
AND 2.355 * g.sigma_balmer < 500
AND 2.355 * g.sigma_forbidden < 500
AND s.snmedian_g > 40
AND g.oiii_5007_flux > 5
AND (g.oiii_5007_flux / g.oiii_5007_flux_err) > 5
AND g.h_alpha_flux > 5
AND (g.h_alpha_flux / g.h_alpha_flux_err) > 5
AND g.h_beta_flux > 5
AND (g.h_beta_flux / g.h_beta_flux_err) > 5
AND g.nii_6584_flux > 5
AND (g.nii_6584_flux / g.nii_6584_flux_err) > 5
\end{sourcecode}

This query defines the sample, gets the spectroscopic information on the needed lines and gets the colours from the WISE catalogue \cite{wise_allsky}. This query resulted in 960 objects. I checked their spectroscopic subclasses and saw that 163 objects are classified as starburst galaxies, 319 as broadline, 71 as AGN broadline, 214 as starforming, 29 as starforming broadline, 84 as AGN, one as starburst broadline, and the rest had no sub-classification.

\subsection{BPT Diagram}

% ----------------------------------------------------------------------------

\section{Part II}

\subsection{Colours from WISE}

I downloaded the WISE colours using the same SQL as the first part \ref{sdss_query}.

\subsection{Classification with WISE Colour-Colour Diagram}



\subsection{Analysing a Particular Object}



% ----------------------------------------------------------------------------

\bibliographystyle{natbib}
\bibliography{references}

\end{document}
