% Template:     Template Auxiliar LaTeX
% Documento:    Archivo principal
% Versión:      7.1.5 (15/03/2021)
% Codificación: UTF-8
%
% Autor: Pablo Pizarro R.
%        Facultad de Ciencias Físicas y Matemáticas
%        Universidad de Chile
%        pablo@ppizarror.com
%
% Sitio web:    [https://latex.ppizarror.com/auxiliares]
% Licencia MIT: [https://opensource.org/licenses/MIT]

% CREACIÓN DEL DOCUMENTO
\documentclass[letterpaper, oneside]{article}

% INFORMACIÓN DEL DOCUMENTO
\def\tituloauxiliar {Photo-Reverberation Mapping}
\def\temaatratar {Tutorial \#5}

\def\autordeldocumento {Nicolás Guerra-Varas}
\def\nombredelcurso {Introduction to AGN}
\def\codigodelcurso {MASS}

\def\nombreuniversidad {University of Belgrade}
\def\nombrefacultad {Faculty of Mathematics}
\def\departamentouniversidad {Departament of Astronomy}
\def\imagendepartamento {img/mass}
\def\imagendepartamentoparams {height=1.9cm}
\def\localizacionuniversidad {Belgrade, Serbia}

% EQUIPO DOCENTE
\def\equipodocente {
	\textbf{Nicolás Guerra-Varas} \\
	Professor Dragana Ili\'c \\
	Tutor Isidora Jankov \\
    \today \\
}

% IMPORTACIÓN DEL TEMPLATE
\input{template}

% INICIO DE PÁGINAS
\begin{document}
	
% CONFIGURACIÓN DE PÁGINA Y ENCABEZADOS
\templatePagecfg

The goal of this tutorial was to learn photo-reverberation mapping (photo-RM) methods and techniques, and to get introduce to tools that simulate active galactic nuclei (AGN) light-curves and perform photo-RM. And then, to apply these methods to real observational data.

% ----------------------------------------------------------------------------

\section*{Task 1}

\section{Simulating Light-Curves}

In the \texttt{photRM.py} module, there are functions implemented to generate artificial AGN light curves. The \texttt{lc\_two\_bands} function generates them and returns light-curves that are ready for photo-RM. There are two main components for these light-curves \cite{Kovacevic_2021, Kelly_2009}:
\begin{itemize}
	\item The $X$ band, which covers only the continuum. This one is generated with a Damped random walk (DRW) process %\cite{Kovacevic_2021, Kelly_2009}
	, which is able to describe optical thermal emission of the accretion disk.
	
	\item The $Y$ band, which covers emission lines and its surrounding continuum. It is modeled as described in \cite{Jankov_2022, Chelouche_Daniel_2012}. The emission line response curve is obtained by convolving the $X$ band light curve with a Gaussian kernel, which mean and standard deviation depend on the radius of the broad line region. Then this is summed up with another pure continuum curve with appropiate realistic weights.
\end{itemize}

I generated three pairs of light curves with $\log L = $ 43, 44 and 45 $L_{\odot}$ respectively. They are 5000 data points long, have redshift $z = 0.1$, have an oscillatory signal with an amplitude of 0.14 magnitudes, noise of a factor of 0.00005, and a random time-lag. In Figures \ref{fig:art_lcs_L43}, \ref{fig:art_lcs_L44} and \ref{fig:art_lcs_L45}, I plotted the first 1000 detections of these light-curves. The time-lags printed by the \texttt{lc\_two\_bands} function are 9.86, 33.65 and 114.82 days respectively.

\begin{figure}[h]
	\centering
	\includegraphics[width=0.8\textwidth]{../lc_plots/art_lcs_logL43.pdf}
	\caption{Artificial light-curve of an AGN with $\log L = 43 L_{\odot}$ and a time-lasg of 9.86 days.}
	\label{fig:art_lcs_L43}
\end{figure}

\begin{figure}[p]
	\centering
	\includegraphics[width=0.8\textwidth]{../lc_plots/art_lcs_logL44.pdf}
	\caption{Artificial light-curve of an AGN with $\log L = 44 L_{\odot}$ and a time-lag of 33.65 days.}
	\label{fig:art_lcs_L44}
\end{figure}

\begin{figure}[p]
	\centering
	\includegraphics[width=0.8\textwidth]{../lc_plots/art_lcs_logL45.pdf}
	\caption{Artificial light-curve of an AGN with $\log L = 45 L_{\odot}$ and a time-lag of 114.82 days.}
	\label{fig:art_lcs_L45}
\end{figure}



\section{Estimating Time-Lags}

Then, I estimated the time-lags of each pair of light-curves. For this, I used \texttt{pyzdcf} \cite{pyzdcf_docs}, a python implementation of a simpler version of \texttt{PLIKE}, a Fortran code \cite{Alexander_2013, tal_alexander_software}. The results are plotted in Figures \ref{fig:ccf_art_lcs_L43}, \ref{fig:ccf_art_lcs_L44} and \ref{fig:ccf_art_lcs_L45}. The estimations closely match the time-lag posted by the \texttt{lc\_two\_bands} function. They are 9.0, 33.0 and 115.0 days respectively. Thus, these are very good results. However, the accuracies and errors are lost in the \texttt{pyzdcf} code.

\begin{figure}[h]
	\centering
	\includegraphics[width=0.8\textwidth]{../CCF_plots/x_y_bands_43_final.pdf}
	\caption{CCF of the artificial light-curve with $\log L = 43 L_{\odot}$}.
	\label{fig:ccf_art_lcs_L43}
\end{figure}

\begin{figure}[p]
	\centering
	\includegraphics[width=0.8\textwidth]{../CCF_plots/x_y_bands_44_final.pdf}
	\caption{CCF of the artificial light-curve with $\log L = 44 L_{\odot}$.}
	\label{fig:ccf_art_lcs_L44}
\end{figure}

\begin{figure}[p]
	\centering
	\includegraphics[width=0.8\textwidth]{../CCF_plots/x_y_bands_45_final.pdf}
	\caption{CCF of the artificial light-curve with $\log L = 45 L_{\odot}$.}
	\label{fig:ccf_art_lcs_L45}
\end{figure}

I note that the more luminous AGN have a larger time-lag (see Figure \ref{fig:art_lum_vs_lag}). This can be explained with the size or supermassive black hole (SMBH) mass versus luminosity relation \cite{Kaspi_2007, Bentz_2009}:
\begin{align}
	R_{BLR} \propto L^{\alpha}
\end{align}

[PHYSICS OF THE THING]

Different authors have estimated $\alpha$. For example, \cite{Kaspi_2005} obtained an estimate of $0.665 \pm 0.069$ and \cite{Bentz_2009} obtained one equal to $0.519 + 0.063 - 0.066$. I plotted this relation to see how it looks like for these artificial light-curves in Figure \ref{fig:art_lum_vs_lag}. None of these slopes fit the artificial light curves, so by trial and error I found that a slope of 0.340 somewhat fits them. The difference might be due to the fact that these light-curves are artificial, or perhaps a redshift correction should be done first.

\begin{wrapfigure}{L}{0.48\textwidth}
	\begin{center}
		\includegraphics[width=0.8\textwidth]{../other_plots/artificial_L_vs_timelag.pdf}
	\end{center}
	\caption{Luminosity versus time-lags for the artificial light-curves.}
	\label{fig:art_lum_vs_lag}
\end{wrapfigure}


\section{Gapped Light-Curves}

\begin{figure}[h]
	\centering
	\includegraphics[width=0.8\textwidth]{../artificial_x_y_5days.pdf}
	\caption{CCF of the artificial light-curve with $\log L = 43 L_{\odot}$ with detections every five days.}.
	\label{fig:ccf_art_lc_L43_5days}
\end{figure}

\begin{figure}[h]
	\centering
	\includegraphics[width=0.8\textwidth]{../artificial_x_y_month.pdf}
	\caption{CCF of the artificial light-curve with $\log L = 43 L_{\odot}$ with detections in a pattern of every day for one month and then a gap of a month.}.
	\label{fig:ccf_art_lc_L43_month}
\end{figure}

\begin{figure}[h]
	\centering
	\includegraphics[width=0.8\textwidth]{../artificial_x_y_lgaps.pdf}
	\caption{CCF of the artificial light-curve with $\log L = 43 L_{\odot}$ with detections in a pattern of three month of observations every day, followed by six months of observations with a frequency of once per month and then a gap of three months.}.
	\label{fig:ccf_art_lc_L43_lgaps}
\end{figure}

% ----------------------------------------------------------------------------

\section*{Task 2: NGC 4395}

% \cite{Edri_2012}


% ----------------------------------------------------------------------------

\bibliographystyle{natbib}
\bibliography{references}

\end{document}
