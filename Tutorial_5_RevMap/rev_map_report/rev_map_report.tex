% Template:     Template Auxiliar LaTeX
% Documento:    Archivo principal
% Versión:      7.1.5 (15/03/2021)
% Codificación: UTF-8
%
% Autor: Pablo Pizarro R.
%        Facultad de Ciencias Físicas y Matemáticas
%        Universidad de Chile
%        pablo@ppizarror.com
%
% Sitio web:    [https://latex.ppizarror.com/auxiliares]
% Licencia MIT: [https://opensource.org/licenses/MIT]

% CREACIÓN DEL DOCUMENTO
\documentclass[letterpaper, oneside]{article}

% INFORMACIÓN DEL DOCUMENTO
\def\tituloauxiliar {Photo-Reverberation Mapping}
\def\temaatratar {Tutorial \#5}

\def\autordeldocumento {Nicolás Guerra-Varas}
\def\nombredelcurso {Introduction to AGN}
\def\codigodelcurso {MASS}

\def\nombreuniversidad {University of Belgrade}
\def\nombrefacultad {Faculty of Mathematics}
\def\departamentouniversidad {Departament of Astronomy}
\def\imagendepartamento {img/mass}
\def\imagendepartamentoparams {height=1.9cm}
\def\localizacionuniversidad {Belgrade, Serbia}

% EQUIPO DOCENTE
\def\equipodocente {
	\textbf{Nicolás Guerra-Varas} \\
	Professor Dragana Ili\'c \\
	Tutor Isidora Jankov \\
    \today \\
}

% IMPORTACIÓN DEL TEMPLATE
\input{template}

% INICIO DE PÁGINAS
\begin{document}
	
% CONFIGURACIÓN DE PÁGINA Y ENCABEZADOS
\templatePagecfg

The goal of this tutorial was to learn photo-reverberation mapping (photo-RM) methods and techniques, and to get introduce to tools that simulate active galactic nuclei (AGN) light-curves and perform photo-RM. And then, to apply these methods to real observational data.

% ----------------------------------------------------------------------------

\section*{Task 1}

\section{Simulating Light-Curves}

In the \texttt{photRM.py} module, there are functions implemented to generate artificial AGN light curves. The \texttt{lc\_two\_bands} function generates them and returns light-curves that are ready for photo-RM. There are two main components for these light-curves:
\begin{itemize}
	\item The $X$ band, which covers only the continuum. This one is generated with a Damped random walk (DRW) proccess \cite{Kovacevic_2021} \cite{Kelly_2009}, which is able to describe optical thermal emission of the accretion disk.
	
	\item The $Y$ band, which covers emission lines and its surrounding continuum. It is modeled as described in \cite{Jankov_2022} \cite{Chelouche_Daniel_2012}. The emission line response curve is obtained by convolving the $X$ band light curve with a Gaussian kernel, which mean and standard deviation depend on the radius of the broad line region. Then this is summed up with another pure continuum curve with appropiate realistic weights.
\end{itemize}

I generated three pairs of light curves with $\log L = $ 43, 44 and 45 $L_{\odot}$ respectively. They are 5000 data points long, have redshift $z = 0.1$, have an oscillatory signal with an amplitude of 0.14 magnitudes, noise of a factor of 0.00005, and a random time-lag. In Figures \ref{fig:art_lcs_L43}, \ref{fig:art_lcs_L44} and \ref{fig:art_lcs_L45}, I plotted the first 1000 detections of these light-curves.

\begin{figure}[h]
	\centering
	\includegraphics[width=0.8\textwidth]{../art_lcs_logL43.pdf}
	\caption{Artificial light-curve of an AGN with $\log L = 43 L_{\odot}$}.
	\label{fig:art_lcs_L43}
\end{figure}

\begin{figure}[h]
	\centering
	\includegraphics[width=0.8\textwidth]{../art_lcs_logL44.pdf}
	\caption{Artificial light-curve of an AGN with $\log L = 44 L_{\odot}$.}
	\label{fig:art_lcs_L44}
\end{figure}

\begin{figure}[h]
	\centering
	\includegraphics[width=0.8\textwidth]{../art_lcs_logL45.pdf}
	\caption{Artificial light-curve of an AGN with $\log L = 44 L_{\odot}$.}
	\label{fig:art_lcs_L45}
\end{figure}



\section{Estimating Time-Lags}



\section{Gapped Light-Curves}



% ----------------------------------------------------------------------------

\section*{Task 2: NGC 4395}

\cite{Edri_2012}


% ----------------------------------------------------------------------------

\bibliographystyle{natbib}
\bibliography{references}

\end{document}
