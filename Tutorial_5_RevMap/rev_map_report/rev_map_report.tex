% Template:     Template Auxiliar LaTeX
% Documento:    Archivo principal
% Versión:      7.1.5 (15/03/2021)
% Codificación: UTF-8
%
% Autor: Pablo Pizarro R.
%        Facultad de Ciencias Físicas y Matemáticas
%        Universidad de Chile
%        pablo@ppizarror.com
%
% Sitio web:    [https://latex.ppizarror.com/auxiliares]
% Licencia MIT: [https://opensource.org/licenses/MIT]

% CREACIÓN DEL DOCUMENTO
\documentclass[letterpaper, oneside]{article}

% INFORMACIÓN DEL DOCUMENTO
\def\tituloauxiliar {Photo-Reverberation Mapping}
\def\temaatratar {Tutorial \#5}

\def\autordeldocumento {Nicolás Guerra-Varas}
\def\nombredelcurso {Introduction to AGN}
\def\codigodelcurso {MASS}

\def\nombreuniversidad {University of Belgrade}
\def\nombrefacultad {Faculty of Mathematics}
\def\departamentouniversidad {Departament of Astronomy}
\def\imagendepartamento {img/mass}
\def\imagendepartamentoparams {height=1.9cm}
\def\localizacionuniversidad {Belgrade, Serbia}

% EQUIPO DOCENTE
\def\equipodocente {
	\textbf{Nicolás Guerra-Varas} \\
	Professor Dragana Ili\'c \\
	Tutor Isidora Jankov \\
    \today \\
}

% IMPORTACIÓN DEL TEMPLATE
\input{template}

% INICIO DE PÁGINAS
\begin{document}
	
% CONFIGURACIÓN DE PÁGINA Y ENCABEZADOS
\templatePagecfg

The goal of this tutorial was to learn photo-reverberation mapping (photo-RM) methods and techniques, and to get introduce to tools that simulate active galactic nuclei (AGN) light-curves and perform photo-RM. And then, to apply these methods to real observational data.

% ----------------------------------------------------------------------------

\section*{Task 1}

\section{Simulating Light-Curves}

In the \texttt{photRM.py} module, there are functions implemented to generate artificial AGN light curves. The \texttt{lc\_two\_bands} function generates them and returns light-curves that are ready for photo-RM. There are two main components for these light-curves \cite{Kovacevic_2021, Kelly_2009}:
\begin{itemize}
	\item The $X$ band, which covers only the continuum. This one is generated with a Damped random walk (DRW) process %\cite{Kovacevic_2021, Kelly_2009}
	, which is able to describe optical thermal emission of the accretion disk.
	
	\item The $Y$ band, which covers emission lines and its surrounding continuum. It is modeled as described in \cite{Jankov_2022, Chelouche_Daniel_2012}. The emission line response curve is obtained by convolving the $X$ band light curve with a Gaussian kernel, which mean and standard deviation depend on the radius of the broad line region. Then this is summed up with another pure continuum curve with appropiate realistic weights.
\end{itemize}

I generated three pairs of light curves with $\log L = $ 43, 44 and 45 $L_{\odot}$ respectively. They are 5000 data points long, have redshift $z = 0.1$, have an oscillatory signal with an amplitude of 0.14 magnitudes, noise of a factor of 0.00005, and a random time-lag. In Figures \ref{fig:art_lcs_L43}, \ref{fig:art_lcs_L44} and \ref{fig:art_lcs_L45}, I plotted the first 1000 detections of these light-curves. The time-lags printed by the \texttt{lc\_two\_bands} function are 9.86, 33.65 and 114.82 days respectively.



\section{Estimating Time-Lags}

Then, I estimated the time-lags of each pair of light-curves. For this, I used \texttt{pyzdcf} \cite{pyzdcf_docs}, a python implementation of a simpler version of \texttt{PLIKE}, a Fortran code \cite{Alexander_2013, tal_alexander_software}.

The method used here is a cross-correlation function (CCF) \cite{White_Peterson_1994}:
\begin{align}
	CCF(\tau) = CCF_{YX}(\tau) - ACF_{X}(\tau),
\end{align}

\noindent where $CCF$ is the cross-correlation function between the $X$ and $Y$ light curves, and $ACF$ is the auto-correlation function, i.e. the cross-correlation of $X$ with itself. The method implements a discrete CF (DCF) \cite{Edelson_Krolik_1988}, which does not need to fill in the gaps to obtain uniformly sampled light curves. Instead, it creates time bins of an appropiate size, and calculates a discrete correlation. Then, the z-DCF method uses a z-transformed DCF \cite{Alexander_1997} over equally populated bins, which makes it more robust against gaps and non-uniformly sampled light curves.

The results are plotted in Figures \ref{fig:ccf_art_lcs_L43}, \ref{fig:ccf_art_lcs_L44} and \ref{fig:ccf_art_lcs_L45}. The estimations closely match the time-lag posted by the \texttt{lc\_two\_bands} function. They are 9.0, 33.0 and 115.0 days respectively. Thus, these are very good results. However, the accuracies and errors are lost in the \texttt{pyzdcf} code.

If I did not have the correct pre-known values of the time-lag, another way of telling if the cross-correlation methos converged properly is finding a single clear gap in the CCF plot. This happens in the three cases, so the estimates are reliable.

I note that the more luminous AGN have a larger time-lag (see Figure \ref{fig:art_lum_vs_lag}). This can be explained with the size or supermassive black hole (SMBH) mass versus luminosity relation \cite{Kaspi_2007, Bentz_2009}:
\begin{align}
	R_{BLR} \propto L^{\alpha}
\end{align}

This relation has to do with the connection between the mass of the SMBH and the velocity dispersion of the bulge. Thus, it is crucially important for accurate estimations of the mass of the SMBH.

Different authors have estimated $\alpha$. For example, \cite{Kaspi_2005} obtained an estimate of $0.665 \pm 0.069$ and \cite{Bentz_2009} obtained one equal to $0.519 + 0.063 - 0.066$. I plotted this relation to see how it looks like for these artificial light-curves in Figure \ref{fig:art_lum_vs_lag}. None of these slopes fit the artificial light curves, so by trial and error I found that a slope of 0.340 somewhat fits them. The difference might be due to the fact that these light-curves are artificial, or perhaps a redshift correction should be done first.

\begin{figure}[h]
	\centering
	\includegraphics[width=0.45\textwidth]{../other_plots/artificial_L_vs_timelag.pdf}
	\caption{Luminosity versus time-lags for the artificial light-curves.}
	\label{fig:art_lum_vs_lag}
\end{figure}


\section{Gapped Light-Curves}

Then, I created three pairs of gapped light curves based on the artificial one with $\log{L} = 43 L_{\odot}$. They have three different cadences:
\begin{enumerate}
	\item One detection every five days.
	
	\item A pattern of a month with detections everyday and then a month of no detections.
	
	\item A pattern of three months of observations eveyrday, then six months of observations once a month and a gap of three months.
	
\end{enumerate}

The correct time-lag for this light curve is 9 days. However, the lag estimated for the one with the first cadence was $15.0 \pm 0.0$ days, for the second cadence it was $9.0 (+ 2.0 - 0.0)$, and for the last one it was $135.4 (+ 36.0 - 10.0)$. I initially expected the first cadence to produce a correct estimate, but even though it is wuite close to the correct lag, it is not fully precise. I think this happens because since the frequency of the detections is more sparse, the flares that occur in the light curve can fit into one of the gaps, artificially producing a larger value of the CCF function. The second cadence did produce the expected 9 days of time-lag, but this estimate now has non-neglectable uncertainties. Then, for the third cadence, the estimate is off by two orders of magnitude and has significant uncertainty, to the point that I would consider the estimate not reliable at all. It has multiple peaks in the ACF plot, and the CCF peak is not very clearly marked.

Based on this, the uncertainties in the CCF method come from non-uniformly sampled light curves, with gaps with different lengths. This creates more unreliability than the detections being further apart from each other but having a consistent frequency.

% ----------------------------------------------------------------------------

\section*{Task 2: NGC 4395}

The NGC 4395 galaxy \cite{simbad_ngc4395} is a nearby low-luminosity Seyfert 1 galaxy with redshift 0.001064. It has a magnitude of 14.55 and $\lambda L_{\lambda} ~ 7 \times 10^{39}$ erg s$^{-1}$ \cite{Abazajian_2009, Edri_2012}. I downloaded the SDSS light curves used by \cite{Edri_2012} from Vizier. These are in the i, r and g filters, corresponding to only continuum, continuum with the H$\alpha$ line, and continuum with the H$\beta$ line respectively. These light-curves are plotted in Figure \ref{fig:ngc4395_lcs} in units of flux and hours. Using \texttt{pyzdcf}, I calculated the time-lag between the i and r filter, and obtained a lag of $4.75 \pm 0.00$ hours.

In their work, \cite{Edri_2012} obtained a lag of $3.46 (+1.59-0.36)$. My estimate lies within the uncertainty range of their results, so both estimates agree. The goal of their work was to test the photometric RM method in contrast to the spectroscopic one. The latter requires spectra with high-resolution and resolved lines, making it expensive and time consuming. The photometric RM method is less accurate, but it is robust, quick and simple to implement \cite{Chelouche_Daniel_2012}. So, it is an incredibly strong tool. For instance, it will be helpful when tracking the immense amount of quasars ($10^7$) that the LSST will find. It could also be used as a mechanism to identify interesting and particular sources, and then proceed to conduct spectroscopic RM on them.

\begin{figure}[h]
	\centering
	\includegraphics[width=0.7\textwidth]{../lc_plots/ngc4395_sdss_lcs.pdf}
	\caption{SDSS light curves for the NGC 4395 AGN (filters i, r and g).}
	\label{fig:ngc4395_lcs}
\end{figure}

\begin{figure}[h]
	\centering
	\includegraphics[width=0.7\textwidth]{../lc_plots/ngc4395_sdss_1st_lc.pdf}
	\caption{First segment of the SDSS light curves for the NGC 4395 AGN (filters i, r and g).}
	\label{fig:ngc4395_1st_lcs}
\end{figure}

\begin{figure}[h]
	\centering
	\includegraphics[width=0.7\textwidth]{../CCF_plots/NGC4395_i_r.pdf}
	\caption{DCCF of the first segment of the SDSS light curves for the NGC 4395 AGN at filters i (continuum) and r (continuum and H$\alpha$).}
	\label{fig:ngc4395_ccf}
\end{figure}

% ----------------------------------------------------------------------------

\bibliographystyle{natbib}
\bibliography{references}

% ----------------------------------------------------------------------------

\section*{Figures}

% \subsection{Artificial Light Curves}

\begin{figure}[h]
	\centering
	\includegraphics[width=0.7\textwidth]{../lc_plots/art_lcs_logL43.pdf}
	\caption{Artificial light-curve of an AGN with $\log L = 43 L_{\odot}$ and a time-lasg of 9.86 days.}
	\label{fig:art_lcs_L43}
\end{figure}

\begin{figure}[h]
	\centering
	\includegraphics[width=0.7\textwidth]{../lc_plots/art_lcs_logL44.pdf}
	\caption{Artificial light-curve of an AGN with $\log L = 44 L_{\odot}$ and a time-lag of 33.65 days.}
	\label{fig:art_lcs_L44}
\end{figure}

\begin{figure}[h]
	\centering
	\includegraphics[width=0.7\textwidth]{../lc_plots/art_lcs_logL45.pdf}
	\caption{Artificial light-curve of an AGN with $\log L = 45 L_{\odot}$ and a time-lag of 114.82 days.}
	\label{fig:art_lcs_L45}
\end{figure}

% \subsection{CCF of Artificial Light Curves}

\begin{figure}[h]
	\centering
	\includegraphics[width=0.8\textwidth]{../CCF_plots/x_y_bands_43_final.pdf}
	\caption{CCF of the artificial light-curve with $\log L = 43 L_{\odot}$}.
	\label{fig:ccf_art_lcs_L43}
\end{figure}

\begin{figure}[h]
	\centering
	\includegraphics[width=0.8\textwidth]{../CCF_plots/x_y_bands_44_final.pdf}
	\caption{CCF of the artificial light-curve with $\log L = 44 L_{\odot}$.}
	\label{fig:ccf_art_lcs_L44}
\end{figure}

\begin{figure}[h]
	\centering
	\includegraphics[width=0.8\textwidth]{../CCF_plots/x_y_bands_45_final.pdf}
	\caption{CCF of the artificial light-curve with $\log L = 45 L_{\odot}$.}
	\label{fig:ccf_art_lcs_L45}
\end{figure}

% \subsection{CCF of Gapped Light Curves}

\begin{figure}[h]
	\centering
	\includegraphics[width=0.8\textwidth]{../CCF_plots/artificial_x_y_5days.pdf}
	\caption{CCF of the artificial light-curve with $\log L = 43 L_{\odot}$ with detections every five days.}.
	\label{fig:ccf_art_lc_L43_5days}
\end{figure}

\begin{figure}[h]
	\centering
	\includegraphics[width=0.8\textwidth]{../CCF_plots/artificial_x_y_month.pdf}
	\caption{CCF of the artificial light-curve with $\log L = 43 L_{\odot}$ with detections in a pattern of every day for one month and then a gap of a month.}.
	\label{fig:ccf_art_lc_L43_month}
\end{figure}

\begin{figure}[h]
	\centering
	\includegraphics[width=0.8\textwidth]{../CCF_plots/artificial_x_y_lgaps.pdf}
	\caption{CCF of the artificial light-curve with $\log L = 43 L_{\odot}$ with detections in a pattern of three month of observations every day, followed by six months of observations with a frequency of once per month and then a gap of three months.}.
	\label{fig:ccf_art_lc_L43_lgaps}
\end{figure}


\end{document}
